\documentclass[a4paper]{article}






%%%%%%%%%%%%%%%%%%%%%%%%%%%%%%%%%%%%%%%


\usepackage[margin=3cm]{geometry} 
\usepackage[utf8]{inputenc}
\usepackage{xeCJK}
\usepackage[usenames, dvipsnames]{color}%可以使用顏色
%\setCJKmainfont{DFKai-SB}
%\setCJKsansfont{DFKai-SB}
%\setCJKmonofont{DFKai-SB}



\title{高等學院及高等學院預備班之問與答}
\author{qq}
\date{}
	

\begin{document}
%\begin{CJK}{UTF8}{bkai}
\maketitle
\tableofcontents
%%%%%%%%%%%%%%%%%%%%%%%%%%%%%%%%%%%%%%%

\section{總論}

\paragraph{問}
高等學院預備班和高等學院分別是什麼?
\paragraph{答}
簡單來說,高等學院是法國特有的教育體制,和大學完全是兩回事。法國高中生畢業之後,可選擇進入大學或者高等學院預備班(以下簡稱預備班),其實還有其他選擇,但因人數比例不多而且我也不甚了解,所以不詳談。通常每年只有約10percnet的高中畢業生,有辦法進入預備班就讀,由此可見進入預備班並不是件容易的事情。一般來說,預備班為期兩年,二年級末參加高等學院入學考試,決定未來升學就業方向,這考試大約從四月份開始一直到七月中結束。
\paragraph{問}
預備班和一般大學的差別在哪裡?
\paragraph{答}
相對於一般大學,預備班對學生素質的要求很嚴格。之前認識幾個在法國讀一般大學的朋友,他們說大學是社區化的,基本上只要通過高中畢業會考(BAC,只有分考過與不過兩種,而且大部分情況下分數不重要)就可以去讀了,而且可以選擇任何你想讀的科系。不過話雖如此,在大學期間仍會有考試篩選,品質基本上還是有把關,只是可能沒有預備班這麼嚴格而已。
  但是預備班篩人的層次又更麻煩了,一年級升上二年級時,除了分組分得更細之外,還有能力分班,從高到低依次為兩星、一星、無星班,學校甚至會把成績太差的同學踢走。不過也有些預備班學生自認到程度不足,自願轉到程度較差的學校或者到一般大學去讀,也有人發現性向不合而轉學轉組的。


\section{高等學院預備班}
\paragraph{問}
你讀的什麼系?
\paragraph{答}
呃,預備班應該不能說是分系啦,比較像是分組。像我的組叫做「數學物理組」,主要就是讀數學和物理。
\paragraph{問}
數學課和物理課在上什麼?
\paragraph{答}
恩,好問題。我也覺得數學課和物理課這兩個名字聽起來很籠統,那我講講一年級的上課內容好了,二年級的東西以後再說。
  數學分為兩個部分,代數和分析,講的東西大概是高等微積分、線性代數、拓撲之類的,總之就是比較理論、比較抽象的數學,但數學課的訓練方式還挺嚴謹的。數學考試都是證明題,通常是把一個定理拆成數十個小題,然後一步一步引導我們作答,進而證明這個定理,各小題之間看似毫無關係,但實際上是一題牽連著一題的,而證明過程更是要一步步嚴謹的寫下來,最忌諱無中生有跑出莫名其妙、毫無根據的東西,證明過程不清楚、不完整就等於沒寫。
  物理的話大概就是像普物一樣,高中物理加上微積分,不過我們一年級是沒有把高中物理的所有範圍都講完啦,像是轉動慣量的部分就完全沒有提到。另外有一點值得一提的就是,他們物理的紀錄方法都用向量很嚴謹的表達,比較不會錯亂,至少和我高中物理學起來的感覺是如此。雖然一開始不怎麼習慣,甚至覺得這樣很麻煩,不過久了以後發現這樣的紀錄方式比較有系統,而且在解複雜問題的時候才不會一團亂。
  然後物理課中有夾雜著少許的化學課,不過化學課真的很無聊,上的東西可能比高中化學還簡單些,所以這就沒什麼好提的了。
\paragraph{問}
那有沒有別的課?像是語文方面的課?
\paragraph{答}
當然有英文課啦。(小聲)不過大部分的課我都翹掉了。其實英文課沒有在上什麼,每次上課老師就發英文或法文的文章下來,叫我們念,然後做翻譯,這種上課方式其實還挺無聊的。而且我們老師不太會教,所以上課班上永遠都是鬧哄哄的,認真聽課也不是,做自己的事情也無法專心,陪同學起鬨鬧老師還挺幼稚無聊的,所以乾脆就翹掉自己讀英文還比較實際點,不過就頂多加強閱讀能力和單字量而已,英法之間翻譯的練習等法文能力進步之後自然就會比較熟練了吧。
  每兩週有一次英文口試,個人認為這還不錯,至少有機會講一下英文,並在滿是法文單字的腦中嘗試撈出幾個英文單字來用。說真的,一年下來有把英文口語能力找回一些,然後表達有稍微流暢點,不過進步空間還是很大就是了。
  除此之外還有法文課,每年會有個課程綱要(例如2008年「我的謎」、2009年「金錢」)以及三本相關文學作品,可能是戲劇、小說、自傳之類的東西,上課時老師則對三本指定閱讀的書籍進行講解,就文學和哲學的方面來探討,上課時還會講一些作者的生活年代、歷史背景之類的,大概可以類比成高中國文課。上課基本上聽不太懂老師在說什麼,雖然講話速度不快,不過用字比較艱深,再加上聽到歷史這種東西我的腦袋就負荷不了…所以整年的法文課過得頗辛苦。
\paragraph{問}
你們上課用什麼課本?
\paragraph{答}
上課並沒有標準課本,要自己做筆記,有時候抄黑板,有時候是用聽寫的方式。不過坊間有出類似參考書之類的東西,還有習題本,不過價錢不怎麼便宜就是了,一本少則一千台幣,多則三千,所以不會想買。重點是買了也沒時間看。
\paragraph{問}
既然參考書這麼貴,該不會是官商勾結吧?
\paragraph{答}
(奸笑)嘿嘿,你覺得勒?

\section{高等學院入學考試}
\paragraph{問}
可不可以說明一下高等學院的入學考試?
\paragraph{答}
有些高等學院是獨立招生,有些是聯合招生,所以有些考試是多個學校通用,有些只有一所學校適用。像我的「數學物理組」有六項考試可以報名,其中包含的學校應該有一百所,不過大部分學校招收的學生人數都不多,而且所學的東西也比較專門。
\paragraph{問}
那要考哪些科目?
\paragraph{答}
數學、物理、資訊、英文、法文,有些學校還有中文可以考。
\paragraph{問}
考試的方式是怎樣?
\paragraph{答}
考試分為筆試和口試,筆試過了方可參加口試。
\paragraph{問}
口試如何考?
\paragraph{答}
在學校,每週都會有一次數學口試,兩週一次物理和英文口試,一年三次法文口試。數學、物理的口試是這樣的:老師會給題目,接著在黑板上作答,若題目太難會適時地提供一些提示,然後有了答案之後,再把思考過程和計算過程告訴老師。英文口試中,有20分鐘準備一篇從報章雜誌剪下來的文章,然後再來的20分鐘是報告時間,要說明這篇文章的摘要、分析文章結構,再加上個人見解;法文口試也頗類似,只是發下來的文章是一篇法文和哲學或文學有關的文章,也是要做摘要然後加上自己的看法。
\paragraph{問}
如果沒考上要怎麼辦?
\paragraph{答}
很簡單,重讀一年就好了;若又沒考上,就繼續重讀,直到考上為止。現實是殘酷的。


\section{其他}
\paragraph{問}
吃飯問題怎麼解決?
\paragraph{答}
週一到週五的三餐,及週六的早餐、中餐學校都有提供,因此只有週末需要自己下廚煮飯,當然也可以去外頭覓食,不過看到價格就飽了。
\paragraph{問}
聽說法國菜很好吃?所以平常都吃什麼?
\paragraph{答}
引用同學對我說過的一句話,「你不能因為平常在學校吃,就以為法國料理是這個樣子,這根本就不算是料理!」由此可見法國人對學校餐廳的伙食還挺有意見的,雖然個人認為沒有特別好吃,可是拿來裹腹還是綽綽有餘啦。
  學校最常出現的主食大概就是馬鈴薯了,蒸的、煮的、磨成泥的、或者炸的,其次是義大利麵、小麥粒(couscous,配上醬超好吃的)或者是飯(和我們認知中的飯多少有些差距),有時候是蔬菜泥,或是燉煮到爛掉的蔬菜,甚至還有跼烤花椰菜。肉類的話各個種類輪流出現,比較特別的是法國人吃兔肉,除了偶爾有不怎麼新鮮的魚,以及幾乎是生的或是硬到咬不動的牛肉之外,大致上還可以。
  但說真的他們蔬菜水果種類和臺灣比起來少太多了,所以在台灣要多吃一點蔬菜水果。
\paragraph{問}
那外面餐廳如何?
\paragraph{答}
有到外面餐廳吃過三、四次,不過感覺沒有特別好吃,除了餐廳布置的還不錯、氣氛很好很適合聊天之外,也沒有什麼特別的,不如回宿舍自己煮,物美價廉經濟實惠。不過有一次印象特別深刻,同學爸爸帶我到海邊的一家餐廳吃海鮮,雖然那時候根本不知道自己吃的東西是什麼,只知道吃起來還不錯,但是一個人一餐要1500台幣…
\paragraph{問}
有參加什麼社團?
\paragraph{答}
法國人好像沒有那麼熱中社團,學校應該有十來個社團,不過大部分都快倒閉的樣子,除了管弦樂隊兩、三個月有一次公演之外,其他社團根本不知道在做什麼。
\paragraph{問}
平常做什麼運動?
\paragraph{答}
假日下午偶爾打個籃球,平常晚上沒事做揪人去公園慢跑,有時候下午會去學校附近游泳,大概就這樣吧。喔,還有,走路代替地鐵也是很好的運動。
\paragraph{問}
除此之外還有其他的休閒活動嗎?
\paragraph{答}
參觀博物館,巴黎博物館密度好高,隨便走走都會碰到。今年(2009)八月起,法國博物館對26歲以下在歐盟地區的居民或持有長期居留證的外國人開放免費參觀。此外,有時候巴黎也會有有趣的活動,像是博物館之夜、夏至音樂節之類的,還挺有趣的。

%\end{CJK}
\end{document}
